\chapter*{Abstract}
\markboth{ABSTRACT}{ABSTRACT}

A theory of polarized electron reflection, at the surface of a
magnetic material, is devised, predicting that the rate of electrons
leaving the surface will be
\begin{equation}
G = \left(\frac{e^2V^2}{16E_b^2\cos^4I}+\frac{e^2\hbar^2B^2}{192m_e^2E_b^2\cos^4I}\right)\frac{F}{e}\pnc{,}
\end{equation}
and that the reflected beam's polarization will be
\begin{equation}
P =
-\frac{4e^2\hbar{}m_eVB}{12e^2m_e^2V^2+e^2\hbar^2B^2}\pnc{.}
\end{equation}
$V$ represents the electrostatic potential inside the material, $B$
the Weiss field inside the material, $I$ the angle of incidence of the
electron beam, $E_b$ each incident electron's kinetic energy, and
$F/e$ the rate of electron incidence.  The equations represent a
lowest order Taylor expansion in $\frac{1}{E_b}$, which, for cobalt
samples, renders them applicable as long as $E_b \gg{}
\sim{}700\ut{meV}$.  They also involve an assumption that that the
topmost layer of the sample has a thickness $\gg{} 1\ut{nm}$.

Measurements are presented of spin-correlated electron arrival rates
at a Mott-polarimeter's two detectors, from the reflected electron beam
from $Co/Cu(001)$, as a function of cobalt thickness, incident energy,
and incident intensity.  Example results are displayed graphically in
chapter \ref{results}.  They are analysed using three distinct
methodologies.
\begin{enumerate}
\item{}Visual inspection of the data is used to make qualitative
  suggestions about future directions for modelling of electron
  reflection processes at magnetic surfaces.  A
  magnetization-dependent systematic error is discovered in electron
  arrival rates at the polarimeter's detectors, and is tentatively
  attributed to the deflection of the electron beam by a stray
  magnetic field, from the sample, or from some part of the sample
  holder.  This raises the possibility, for the future, of using a
  reflected electron beam's spatial deflection, rather than its spin
  polarization, to characterize magnetic samples.
\item{}A traditional estimator of the Mott asymmetry, associated with
  the reflected polarization, is calculated for each combination of
  film thickness, incident energy, and incident intensity.  Results
  are displayed graphically in section
  \ref{traditional-estimation-section}.  The polarizations of most
  reflected beams from cobalt films are clearly non-zero, indicating
  that the experiment has successfully detected the cobalt's Weiss
  field.  Quantitative determination of this polarization is, however,
  rather imprecise, and there is a puzzling non-zero asymmetry from
  the bare copper surface.
\item{}It is argued that the accuracy of the traditional estimator may
  be compromised by its non-linearity in the electron arrival rates;
  therefore, Bayesian inference is used to estimate the parameters in
  two adaptive models, based on the above theory, and to estimate
  relative probabilities for these models, equipped with the data.
  One model has a non-zero Weiss field in the films, the other does
  not.  Results of the parameter estimation are displayed graphically
  in sections \ref{adaptive-models} and \ref{further-conclusions};
  quantitative estimation of the sample properties is, as for the
  traditional estimator, rather imprecise.  The data are found to rule
  out a null hypothesis, in which the cobalt's Weiss field is zero,
  very strongly, leaving it with a posterior probability of
  \input{null-posterior}, indicating firmly that the experiment has
  successfully detected the cobalt's Weiss field.
\end{enumerate}

These measurements were preceded by a process (sections
\ref{autumn-2001-experiments}, \ref{july-2002-experiments}) of trial
and error, in which attempted measurements of the spin polarization of
reflected electron beams revealed sources of systematic error, which
were addressed by adaptations to the polarimeter, and to the
experimental technique.
